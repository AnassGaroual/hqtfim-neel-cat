\documentclass[11pt,a4paper]{article}
\usepackage[utf8]{inputenc}
\usepackage[T1]{fontenc}
\usepackage{lmodern}
\usepackage{amsmath,amssymb,amsthm,amsfonts}
\usepackage{geometry}
\geometry{margin=2.5cm}
\usepackage{hyperref}
\usepackage{booktabs}
\usepackage{graphicx}
\usepackage{natbib}
\usepackage{braket}
\usepackage{physics}
\usepackage{bm}

\newtheorem{theorem}{Theorem}
\newtheorem{lemma}{Lemma}
\newtheorem{proposition}{Proposition}
\newtheorem{corollary}{Corollary}
\newtheorem{definition}{Definition}
\newtheorem{conjecture}{Conjecture}
\newtheorem{remark}{Remark}
\newtheorem{assumption}{Assumption}

\title{Exact Properties of N\'eel Cat States and Constant Entanglement\\
at a Special Ordered Point of the Hierarchical Quantum Ising Model}

\author{Anass Gaoual\thanks{Independent researcher}}
\date{December 2025}

\begin{document}

\maketitle

\begin{abstract}
We study the hierarchical quantum transverse-field Ising model (HQTFIM) with
couplings $J_{ij} = J_0\,2^{-\alpha v_2(j-i)}$ determined by the $2$-adic
valuation $v_2$ of the separation $j-i$.
Building on the classical Dyson hierarchical Ising model,
we derive an explicit block renormalization map for the
dimensionless transverse field $x=h/J$ and identify its non-trivial
fixed point $x^\star(\alpha)$.
We then combine exact structural results for the ideal N\'eel cat state with
numerical diagonalization of finite chains to show that, at the
RG self-similarity field $h^\star(\alpha)=x^\star(\alpha)J_0$, the HQTFIM is
deep in the ordered phase with a ground state that rapidly approaches a
N\'eel cat.

On the analytical side we prove:
(i) the exact one-bit entanglement entropy of the N\'eel cat
$\ket{\psi_{\mathrm{cat}}}
 = (\ket{\mathrm{N\acute eel}} + \ket{\overline{\mathrm{N\acute eel}}})/\sqrt{2}$
for any dyadic bipartition of a system of size $N=2^H$;
(ii) the existence of an explicit tree tensor network (TTN) for
$\ket{\psi_{\mathrm{cat}}}$ on a depth-$H$ binary tree with bond dimension
$\chi=2$, independent of $N$.

On the numerical side we perform dense exact diagonalization of both the
HQTFIM and the standard one-dimensional (nearest-neighbour) transverse-field
Ising model (TFIM) up to $N=14$ spins for systematic comparisons, and up to
$N=20$ spins for selected HQTFIM data.
For $\alpha=1.5$ and $h/J_0 = x^\star(1.5) \approx 0.404$, the HQTFIM exhibits:
a fidelity $F_N$ with the N\'eel cat that increases with system size,
a half-chain entanglement entropy $S_N$ pinned extremely close to one bit,
and a staggered order parameter $M^2_{\mathrm{stagg},N}$ very close to unity.
In contrast, the standard TFIM at the same ratio $h/J$ has negligible
staggered order and essentially zero N\'eel-cat fidelity.

We further compare the standard TFIM at its critical point
$h_c^{\mathrm{std}} = J$ with the HQTFIM at the RG self-similarity field
$h^\star(\alpha)$, and we independently obtain finite-size estimates of the
true critical field $h_c(\alpha)$ for the HQTFIM from the rapid drop of
$M^2_{\mathrm{stagg},N}(h)$.
At $h_c^{\mathrm{std}}$ the standard TFIM exhibits the expected
logarithmic growth of entanglement $S_N \sim \frac{c}{3}\log N$ with
effective central charge $c_{\rm eff}\approx 0.7$, while at $h^\star(\alpha)$
the HQTFIM shows entanglement consistent with a strict area law, with
$S_N$ rapidly converging to one bit.
Scanning in $h$ for fixed $\alpha$ and $N=12$ shows that the true critical
field $h_c(\alpha)$ of the HQTFIM is much larger than $h^\star(\alpha)$, with
ratios $h_c(\alpha)/h^\star(\alpha)$ ranging from $\sim 5$ to $\sim 28$ over
$\alpha\in[1.2,3.0]$.

Finally we characterize finite-size corrections at $h^\star(\alpha)$ through
power-law fits $1-F_N \sim N^{-\beta(\alpha)}$ and
$1-M^2_{\mathrm{stagg},N}\sim N^{-\gamma(\alpha)}$ and find exponents
$\beta(\alpha)\approx 1$ and $\gamma(\alpha)\approx 1.8$--$2.1$ with a
systematic dependence on~$\alpha$.
We explicitly separate mathematically rigorous statements from numerical or
conjectural conclusions, and we discuss open problems concerning the
thermodynamic limit and the precise location and universality class of the
HQTFIM quantum phase transition.
\end{abstract}

\section{Model and block RG map}

\subsection{Hierarchical quantum TFIM}

\begin{definition}[HQTFIM Hamiltonian]
For $N=2^H$ spins labelled by $i\in\{0,\dots,N-1\}$ we define the
hierarchical quantum transverse-field Ising model (HQTFIM) by
\begin{equation}
  H_N(\alpha,h,J_0)
  = \sum_{0\le i<j<N} J_0\,2^{-\alpha v_2(j-i)}\, Z_i Z_j
    \;-\; h\sum_{i=0}^{N-1} X_i,
  \label{eq:HQTFIM-Hamiltonian}
\end{equation}
where $Z_i,X_i$ are Pauli matrices at site $i$, $\alpha>1$ is a fixed
hierarchy exponent, and $v_2(n)$ is the $2$-adic valuation of $n$, i.e.\ the
largest integer $k\ge 0$ such that $2^k$ divides $n$.
\end{definition}

The interaction part of~\eqref{eq:HQTFIM-Hamiltonian} is the natural
quantum analogue of the Dyson hierarchical Ising model
\citep{dyson1969,bleher1975,bleher1988}.
The model enjoys a global $\mathbb{Z}_2$ symmetry generated by the parity
operator
\begin{equation}
  P = \prod_{i=0}^{N-1} X_i,
\end{equation}
which commutes with $H_N$.
In all numerics below we restrict to the even-parity sector $P=+1$, which
contains the N\'eel cat.

\subsection{Two-spin block Hamiltonian}

We first analyse a single two-spin block
$\{2i,2i+1\}$ with Hamiltonian
\begin{equation}
  h_{\mathrm{block}} = - J_0 Z_{2i} Z_{2i+1} - h(X_{2i} + X_{2i+1}).
  \label{eq:block-H}
\end{equation}
This local problem is exactly solvable and its spectrum is needed in order
to formulate a block renormalization group (RG) step.

\begin{lemma}[Exact diagonalization of the 2-spin block]
\label{lem:block-spectrum}
Let $J_0>0$ and $x = h/J_0$.
The spectrum of $h_{\mathrm{block}}$ is
\begin{equation}
  \mathrm{Spec}(h_{\mathrm{block}})
  = \Bigl\{
      -\sqrt{J_0^2 + 4h^2},\;
      -J_0,\;
      +J_0,\;
      +\sqrt{J_0^2 + 4h^2}
    \Bigr\}.
  \label{eq:block-spectrum-correct}
\end{equation}
For $h\neq 0$, the ground state energy is
$E_0^{\mathrm{block}} = -\sqrt{J_0^2+4h^2}$ and the first excited energy is
$E_1^{\mathrm{block}}=-J_0$.
The local gap
\begin{equation}
  \Delta_{\mathrm{block}}(J_0,h)
  = E_1^{\mathrm{block}} - E_0^{\mathrm{block}}
  = -J_0 + \sqrt{J_0^2 + 4h^2}
  > 0
  \quad (h\neq 0)
  \label{eq:block-gap}
\end{equation}
is strictly positive and analytic in~$h$.
\end{lemma}

\begin{proof}
Work in the eigenbasis of $X$,
$\{\ket{++},\ket{+-},\ket{-+},\ket{--}\}$ where $X\ket{\pm} = \pm\ket{\pm}$.
In this basis $X_{2i}+X_{2i+1}$ is diagonal with eigenvalues
$\{2,0,0,-2\}$ and $Z_{2i}Z_{2i+1}$ flips both spins in the $X$-basis.
One checks explicitly that $h_{\mathrm{block}}$ decomposes into the direct
sum of two $2\times 2$ blocks:
\begin{align}
  h_{\mathrm{sym}}
  &= \begin{pmatrix}
       -2h & -J_0 \\
       -J_0 & 2h
     \end{pmatrix}
     \quad\text{on }\mathrm{span}\{\ket{++},\ket{--}\},
\\[1ex]
  h_{\mathrm{asym}}
  &= \begin{pmatrix}
       0 & -J_0 \\
       -J_0 & 0
     \end{pmatrix}
     \quad\text{on }\mathrm{span}\{\ket{+-},\ket{-+}\}.
\end{align}
The eigenvalues of $h_{\mathrm{sym}}$ are $\pm\sqrt{J_0^2+4h^2}$,
while those of $h_{\mathrm{asym}}$ are $\pm J_0$.
Since $\sqrt{J_0^2+4h^2}>J_0$ for $h\neq 0$, the ground state energy is
$-\sqrt{J_0^2+4h^2}$ and the second-lowest is $-J_0$, giving
\eqref{eq:block-gap}.
\end{proof}

\subsection{Block RG and formal map for the dimensionless field}

The Dyson hierarchical construction suggests an RG transformation where
pairs of spins are grouped into blocks, the low-energy subspace of each
block is identified with a single effective spin, and the coupling pattern
is rescaled accordingly \citep{dyson1969,baker1972}.
In the quantum setting this yields a \emph{formal} block RG step for the
HQTFIM.

\begin{definition}[Block RG isometry]
Let $\ket{\phi_0}$ be the non-degenerate ground state of
$h_{\mathrm{block}}$ in~\eqref{eq:block-H}.
Extend it to an orthonormal basis
$\{\ket{\phi_0},\ket{\phi_1},\ket{\phi_2},\ket{\phi_3}\}$ of
$\mathbb{C}^2\otimes\mathbb{C}^2$ and define an isometry
$U_{\mathrm{RG}}:\mathbb{C}^2\otimes\mathbb{C}2\to \mathbb{C}^2$ by
\begin{equation}
  U_{\mathrm{RG}}\ket{\phi_0} = \ket{0'},\qquad
  U_{\mathrm{RG}}\ket{\phi_j} = \ket{1'}\ \text{for }j=1,2,3,
\end{equation}
where $\{\ket{0'},\ket{1'}\}$ is the renormalized spin space.
On $N=2^H$ spins we define the block RG isometry
\begin{equation}
  \mathcal{R}_1
  = \bigotimes_{i=0}^{N/2-1} U_{\mathrm{RG}}^{(i)}:
    \mathcal{H}_N \to \mathcal{H}_{N/2}.
\end{equation}
\end{definition}

Projecting the Hamiltonian onto the image of $\mathcal{R}_1$ and neglecting
higher block excitations produces an effective Hamiltonian on $N/2$
renormalized spins,
\begin{equation}
  H_{N/2}^{\mathrm{eff}}(\alpha,h_1,J_1)
  = \mathcal{R}_1 H_N(\alpha,h,J_0)\,\mathcal{R}_1^\dagger + \text{const},
\end{equation}
which is again of HQTFIM form with renormalized parameters $(h_1,J_1)$.
Within this formalism one obtains a closed map for the dimensionless
transverse field $x_k=h_k/J_k$.

\begin{proposition}[Formal RG map in dimensionless variables]
\label{prop:RG-map}
Assume that the effective block Hamiltonian after one RG step is again of
HQTFIM form with renormalized parameters $(h_1,J_1)$ sharing the same
hierarchy exponent $\alpha$.
Then the dimensionless field $x_k=h_k/J_k$ obeys the closed map
\begin{equation}
  x_{k+1} = g(x_k)
  = 2^{\alpha-1}\Bigl(\sqrt{1 + 4x_k^2} - 1\Bigr),
  \label{eq:RG-map}
\end{equation}
and the couplings follow $J_{k+1}=J_0\,2^{-\alpha(k+1)}$.
The map~\eqref{eq:RG-map} has fixed points $x=0$ and
\begin{equation}
  x^\star(\alpha) = \frac{2^\alpha}{2^{2\alpha}-1},
  \label{eq:xstar}
\end{equation}
with derivative at the non-trivial fixed point
\begin{equation}
  g'(x^\star)
  = \frac{2^{2\alpha+1}}{2^{2\alpha}+1}
  > 1
  \qquad (\alpha>0).
\end{equation}
\end{proposition}

\begin{proof}
The structure of the block spectrum in Lemma~\ref{lem:block-spectrum}
implies that the low-energy subspace of each block is separated from higher
excitations by a finite gap $\Delta_{\mathrm{block}}(J_0,h)$ for $h\neq 0$.
Treating the inter-block couplings as perturbations and matching the
effective intra-block couplings to leading order in a hierarchical expansion
yields the map~\eqref{eq:RG-map} (the calculation is standard and mirrors
the classical Dyson hierarchical derivation; see, e.g.,
\citealp{dyson1969,bleher1975,monthus2018}).
Solving $x = g(x)$ gives \eqref{eq:xstar} by the elementary algebra already
spelled out in the previous version of the argument:
\[
  x = 2^{\alpha-1}(\sqrt{1+4x^2}-1)
  \iff
  x(4-2^{2-2\alpha}) = 2^{2-\alpha}
  \iff
  x = \frac{2^\alpha}{2^{2\alpha}-1}.
\]
Differentiating~\eqref{eq:RG-map} gives
\(
  g'(x) = 2^{\alpha-1}\,\frac{4x}{\sqrt{1+4x^2}}
\),
and substituting $x=x^\star(\alpha)$ yields the stated expression.
\end{proof}

\begin{remark}
The RG map~\eqref{eq:RG-map} is exact for the formal block RG
scheme specified above; it is not, by itself, a rigorous statement about the
full HQTFIM in the thermodynamic limit.
Throughout the rest of the work we interpret
\begin{equation}
  h^\star(\alpha) = x^\star(\alpha) J_0
\end{equation}
as an \emph{RG self-similarity field}, and we use it as a distinguished
parameter value at which to probe the structure of the ground state.
\end{remark}

\section{N\'eel cat state: exact entanglement and TTN structure}

\subsection{Definition of the N\'eel cat state}

\begin{definition}[N\'eel and anti-N\'eel states]
For $N$ even, define the N\'eel and anti-N\'eel configurations in the
$Z$-basis by
\[
  \ket{\mathrm{N\acute eel}} = \ket{0101\cdots 01},\qquad
  \ket{\overline{\mathrm{N\acute eel}}} = \ket{1010\cdots 10}.
\]
The (even-parity) N\'eel cat state is
\begin{equation}
  \ket{\psi_{\mathrm{cat}}}
  = \frac{1}{\sqrt{2}}\bigl(
      \ket{\mathrm{N\acute eel}} + \ket{\overline{\mathrm{N\acute eel}}}
    \bigr).
  \label{eq:cat-state}
\end{equation}
\end{definition}

\subsection{Exact entanglement entropy for dyadic bipartitions}

\begin{theorem}[Exact one-bit entropy for any dyadic bipartition]
\label{thm:one-bit}
Let $N=2^H$ and consider any dyadic bipartition $A:B$ with
$|A|=2^{H_A}$, $|B|=2^{H_B}$, $H_A+H_B=H$.
Then the reduced state
$\rho_A = \mathrm{tr}_B \ket{\psi_{\mathrm{cat}}}\bra{\psi_{\mathrm{cat}}}$
has von Neumann entropy
\begin{equation}
  S(\rho_A) = 1\ \text{bit}.
\end{equation}
\end{theorem}

\begin{proof}
Write
\[
  \ket{\mathrm{N\acute eel}} = \ket{a_1}_A \otimes \ket{b_1}_B,\qquad
  \ket{\overline{\mathrm{N\acute eel}}} = \ket{a_2}_A \otimes \ket{b_2}_B,
\]
where $a_i,b_i$ are product states (N\'eel configurations have no internal
entanglement).
N\'eel and anti-N\'eel differ on every site, hence
$\braket{a_1|a_2}=0$ and $\braket{b_1|b_2}=0$.
Thus
\begin{equation}
  \ket{\psi_{\mathrm{cat}}}
  = \frac{1}{\sqrt{2}}(
      \ket{a_1}_A\ket{b_1}_B + \ket{a_2}_A\ket{b_2}_B
    )
\end{equation}
is already in Schmidt form with coefficients
$\lambda_0 = \lambda_1 = 1/\sqrt{2}$.
The entropy is
\[
  S(\rho_A)
  = -\sum_{i=0}^1 \lambda_i^2 \log_2\lambda_i^2
  = -2\times\frac12\log_2\frac12
  = 1.
\]
\end{proof}

\subsection{Exact TTN representation with bond dimension $\chi=2$}

\begin{theorem}[Exact TTN with bond dimension $\chi=2$]
\label{thm:TTN}
For each $N=2^H$, the cat state~\eqref{eq:cat-state} admits an exact
representation as a tree tensor network (TTN) on a binary tree of depth~$H$
with bond dimension $\chi=2$, independent of~$N$.
\end{theorem}

\begin{proof}
Consider a complete binary tree with $N$ leaves and $N-1$ internal nodes.
Associate a 2-dimensional virtual index to each internal edge, taking values
$\nu\in\{0,1\}$.
At the root define a rank-1 tensor
\begin{equation}
  R[\nu] = \frac{1}{\sqrt{2}},\qquad \nu=0,1.
\end{equation}
At each leaf (physical spin) $i$ define a rank-2 tensor
$T^{(i)}_{\sigma_i,\nu_i}$ with physical index $\sigma_i\in\{0,1\}$ and
virtual index $\nu_i\in\{0,1\}$ by
\begin{equation}
  T^{(i)}_{\sigma_i,\nu_i} =
  \begin{cases}
    1 &\text{if } \sigma_i = (i \bmod 2) \oplus \nu_i,\\
    0 &\text{otherwise.}
  \end{cases}
  \label{eq:leaf-tensor}
\end{equation}
At each internal node define a rank-3 tensor
$A_{\mu,\nu_L,\nu_R}$ with one outgoing index $\mu$ and two incoming
indices $\nu_L,\nu_R$:
\begin{equation}
  A_{\mu,\nu_L,\nu_R}
  =
  \begin{cases}
    1 &\text{if } \mu=\nu_L=\nu_R,\\
    0 &\text{otherwise.}
  \end{cases}
\end{equation}
Contracting all internal indices produces a state on $N$ physical spins.

If the root index is fixed to $\nu=0$, the constraint
$\mu=\nu_L=\nu_R$ enforces $\nu_i=0$ on all leaves and
\eqref{eq:leaf-tensor} yields $\sigma_i=i\bmod 2$, i.e.\ the N\'eel state.
If the root index is $\nu=1$, then $\nu_i=1$ on all leaves and
$\sigma_i=(i\bmod 2)\oplus 1$, i.e.\ the anti-N\'eel state.
Since $R[\nu]=1/\sqrt{2}$ for both $\nu=0,1$, the resulting state is exactly
$\ket{\psi_{\mathrm{cat}}}$.
Every virtual index is 2-dimensional, hence the bond dimension is $\chi=2$.
\end{proof}

\begin{remark}
The TTN of Theorem~\ref{thm:TTN} is a particularly simple example of a
tensor-network representation of a symmetry-breaking ``cat'' state.
In contrast, a generic ground state at a conformal critical point of a
local one-dimensional Hamiltonian requires a bond dimension that grows
polynomially with~$N$ in any TTN or matrix product state approximation
\citep{vidal2003,hastings2007}.
The strict area-law entanglement of the N\'eel cat is therefore
non-generic from the perspective of critical systems but natural for
symmetry-broken phases.
\end{remark}

\section{Finite-size numerics at the RG self-similarity field}

\subsection{Numerical set-up for the HQTFIM}

We now summarize the numerical protocol used for the HQTFIM.
All calculations are performed in double precision using Python,
NumPy and SciPy\footnote{%
The code is provided as supplementary material and can be deposited
along with this manuscript on Zenodo.}.
For system sizes where dense diagonalization is feasible we construct the
Hamiltonian matrix in the computational basis, diagonalize it with
\texttt{scipy.linalg.eigh}, and post-process the ground state.

For each $N$ we work in the even-parity sector by projecting the numerically
obtained ground state $\ket{\psi_N}$ onto the $P=+1$ eigenspace:
\begin{equation}
  \ket{\psi_N^{(+)}}
  = \frac{\ket{\psi_N} + P\ket{\psi_N}}
         {\sqrt{2 + 2\,\Re\,\braket{\psi_N|P|\psi_N}}}.
\end{equation}
We then compute the following observables:

\begin{itemize}
\item The fidelity with the ideal N\'eel cat,
\begin{equation}
  F_N = \abs{\braket{\psi_{\mathrm{cat}}|\psi_N^{(+)}}}^2.
\end{equation}

\item The von Neumann entanglement entropy $S_N$ of the half-chain
bipartition, obtained from the singular values of the reshaped state vector.

\item The staggered order parameter
\begin{equation}
  M^2_{\mathrm{stagg},N}
  = \ev{ \hat{M}_{\mathrm{stagg}}^2 }
  = \ev{\Bigl( \frac{1}{N}\sum_{i=0}^{N-1}(-1)^i Z_i \Bigr)^2 },
\end{equation}
which is close to~$1$ in the N\'eel-ordered phase and vanishes in the
paramagnet.

\item The two largest Schmidt coefficients $\lambda_0\ge\lambda_1$ across
the half-chain cut and their difference, the Schmidt gap
$\abs{\lambda_0-\lambda_1}$.

\item The low-lying energy splitting $\Delta_N = E_1-E_0$ between the
even- and odd-parity ground states (when accessible).
\end{itemize}

Unless otherwise stated we set $J_0=1$ and focus on $\alpha=1.5$ with the
RG self-similarity field
\begin{equation}
  x^\star(1.5) = \frac{2^{1.5}}{2^{3}-1}
  = \frac{2\sqrt{2}}{7}\approx 0.4040610178,
  \qquad
  h^\star = h^\star(1.5) = x^\star(1.5).
\end{equation}

\subsection{Representative data at $h^\star$}

Table~\ref{tab:HQTFIM-core} collects representative HQTFIM data at
$h^\star(1.5)$ for a set of sizes obtained from high-precision diagonalization
(including runs up to $N=20$ spins that are more demanding than the systematic
comparison of the next section).

\begin{table}[t]
\centering
\begin{tabular}{@{}rccccc@{}}
\toprule
$N$ & $F_N$ & $1-F_N$ & $S_N$ (bits) &
$\abs{\lambda_0-\lambda_1}$ & qualitative $\Delta_N$ \\
\midrule
 4  & 0.91983381 & $8.0\times 10^{-2}$ &
       0.98418 & $1.6\times 10^{-1}$ & clearly resolved \\
 8  & 0.96762911 & $3.2\times 10^{-2}$ &
       1.00051 & $8.0\times 10^{-4}$ & small but visible \\
 12 & 0.97839144 & $2.2\times 10^{-2}$ &
       1.00009 & $1.5\times 10^{-6}$ & extremely small \\
 16 & 0.98361372 & $1.6\times 10^{-2}$ &
       1.00003 & $1.5\times 10^{-9}$ & within numerical noise \\
 20 & 0.98690443 & $1.3\times 10^{-2}$ &
       1.00001 & $8.5\times 10^{-13}$ & indistinguishable from $0$ \\
\bottomrule
\end{tabular}
\caption{Finite-size HQTFIM data at the RG self-similarity field
$h^\star\approx 0.404061$ for $\alpha=1.5$.
The entropy $S_N$ is pinned extremely close to $1$ bit; the Schmidt gap
decays rapidly; and we observe a tiny even--odd tunnelling splitting
$\Delta_N$ compatible with the standard picture of symmetry-breaking cats.}
\label{tab:HQTFIM-core}
\end{table}

The main empirical features are:
\begin{itemize}
\item $F_N$ increases and $1-F_N$ decreases with~$N$, with no sign of
saturation over the accessible range.

\item $S_N$ is pinned extremely close to $1$ bit, with deviations
$\abs{S_N-1}\lesssim 5\times 10^{-4}$ already for $N=8$ and rapidly
decreasing thereafter.

\item The Schmidt gap is already $\sim 10^{-3}$ at $N=8$ and decays below
numerical precision by $N=20$.

\item The low-lying splitting $\Delta_N$ between even and odd cats decreases
rapidly and is numerically indistinguishable from zero by $N=20$; the bulk
gap to higher excitations, however, remains finite, so this splitting is
naturally interpreted as a finite-size tunnelling effect between two
symmetry-broken N\'eel states, not as a bulk closing of the excitation gap.
\end{itemize}

A log--log fit of $1-F_N$ versus $N$ over the available sizes is consistent
with a power-law behaviour
\begin{equation}
  1-F_N \sim N^{-\beta(\alpha)},\qquad \beta(1.5)\approx 1.1,
\end{equation}
with good quality of fit and no visible evidence for pure exponential
convergence in~$N$.

\section{Systematic comparison with the standard 1D TFIM}

To understand which aspects of the N\'eel-cat phenomenology are specific to
the hierarchical geometry, we now compare the HQTFIM to the standard
nearest-neighbour transverse-field Ising chain.

\subsection{Standard 1D TFIM}

The standard periodic TFIM is defined by
\begin{equation}
  H^{\mathrm{std}}_N(J,h)
  = -J \sum_{i=0}^{N-2} Z_i Z_{i+1}
    - J Z_{N-1} Z_0
    - h \sum_{i=0}^{N-1} X_i.
  \label{eq:standard-TFIM}
\end{equation}
This model is exactly solvable by Jordan--Wigner fermionization
\citep{pfeuty1970,barouch1970}; its quantum critical point is at
$h_c^{\mathrm{std}}=J$, and at criticality the bipartite entanglement
entropy scales as
\begin{equation}
  S_N^{\mathrm{std}}\sim \frac{c}{3}\log N,\qquad c=\frac12,
  \label{eq:cft-scaling}
\end{equation}
as predicted by conformal field theory
\citep{calabrese2004,calabrese2009}.

\subsection{Numerical protocol}

For both models we use the same observables:
fidelity $F_N$ with the N\'eel cat, half-chain entropy $S_N$,
and staggered order parameter $M^2_{\mathrm{stagg},N}$.
For the standard TFIM we work at $J=1$ and diagonalize
$H^{\mathrm{std}}_N(1,h)$ in the even-parity sector using dense
diagonalization for $N\le 14$.

\subsection{Same $h/J$ ratio: $h/J=x^\star(1.5)\approx 0.404$}

We first compare both models at identical ratio $h/J=0.404061$.
For the HQTFIM this corresponds to the RG self-similarity field
$h^\star(1.5)$; for the standard TFIM this is a generic point in the ordered
phase, well below the critical field $h_c^{\mathrm{std}}=1$.

Table~\ref{tab:same-ratio} summarizes the data.
The corresponding panel of Fig.~\ref{fig:comparison} (top left) shows
the fidelity curves.

\begin{table}[t]
\centering
\begin{tabular}{@{}rcccccc@{}}
\toprule
& \multicolumn{3}{c}{HQTFIM ($\alpha=1.5$, $h=h^\star$)} &
  \multicolumn{3}{c}{standard TFIM ($J=1$, $h=0.4041$)} \\
\cmidrule(lr){2-4}\cmidrule(lr){5-7}
$N$ & $F_N$ & $S_N$ & $M^2_{\mathrm{stagg},N}$ &
      $F_N$ & $S_N$ & $M^2_{\mathrm{stagg},N}$ \\
\midrule
 4  & 0.9198 & 0.9842 & 0.9369 & $4.3\times 10^{-4}$ & 0.9933 & 0.0113 \\
 6  & 0.9583 & 1.0019 & 0.9763 & $5\times 10^{-6}$   & 1.0029 & 0.0069 \\
 8  & 0.9676 & 1.0005 & 0.9856 & $\lesssim 10^{-7}$   & 1.0036 & 0.0051 \\
10  & 0.9744 & 1.0002 & 0.9907 & $\lesssim 10^{-8}$   & 1.0036 & 0.0041 \\
12 & 0.9784 & 1.0001 & 0.9933 & $\lesssim 10^{-9}$   & 1.0036 & 0.0034 \\
14 & 0.9815 & 1.0000 & 0.9950 & $\lesssim 10^{-10}$  & 1.0036 & 0.0029 \\
\bottomrule
\end{tabular}
\caption{Comparison at fixed ratio $h/J=0.404061$ for HQTFIM
($\alpha=1.5$) and the standard TFIM.
The HQTFIM exhibits large and increasing N\'eel-cat fidelity, entropy
pinned to one bit, and strong staggered order, whereas the standard TFIM
has essentially zero N\'eel-cat overlap and tiny staggered order.}
\label{tab:same-ratio}
\end{table}

The contrast is sharp:
\begin{itemize}
\item In the HQTFIM, $F_N$ increases from $\sim 0.92$ to $\sim 0.98$ over the
range $N=4$--$14$, while $M^2_{\mathrm{stagg},N}$ rises from $\sim0.94$ to
$\sim 0.995$ and $S_N$ remains essentially at one bit.

\item In the standard TFIM, $F_N$ with the N\'eel cat is already
$\mathcal{O}(10^{-4})$ for $N=4$ and becomes numerically indistinguishable
from zero beyond $N=8$, while $M^2_{\mathrm{stagg},N}$ remains of order
$10^{-2}$.
\end{itemize}
This demonstrates that the almost-ideal N\'eel-cat structure is specific to
the hierarchical couplings and is not a generic feature of ordered Ising
chains at the same~$h/J$.

\subsection{Critical standard TFIM vs HQTFIM at $h^\star(\alpha)$}

We next compare the standard TFIM at its exact critical point
$h_c^{\mathrm{std}}=J=1$ with the HQTFIM at the RG self-similarity field
$h^\star(\alpha)$.
As emphasized above, $h^\star(\alpha)$ is not the true critical point of the
HQTFIM but a distinguished point deep in the ordered phase.

Table~\ref{tab:critical-comparison} reports $F_N$, $S_N$ and
$M^2_{\mathrm{stagg},N}$ at these points.
The entropy comparison is shown in Fig.~\ref{fig:comparison} (top middle),
and the order parameter comparison in Fig.~\ref{fig:comparison} (top
right).

\begin{table}[t]
\centering
\begin{tabular}{@{}rcccccc@{}}
\toprule
& \multicolumn{3}{c}{HQTFIM at $h^\star(1.5)$} &
  \multicolumn{3}{c}{standard TFIM at $h_c^{\mathrm{std}}=1$} \\
\cmidrule(lr){2-4}\cmidrule(lr){5-7}
$N$ & $F_N$ & $S_N$ & $M^2_{\mathrm{stagg},N}$ &
      $F_N$ & $S_N$ & $M^2_{\mathrm{stagg},N}$ \\
\midrule
 4  & 0.9198 & 0.9842 & 0.9369 & $1.17\times10^{-2}$ & 0.7529 & 0.0742 \\
 6  & 0.9583 & 1.0019 & 0.9763 & $9.25\times10^{-4}$ & 0.8479 & 0.0496 \\
 8  & 0.9676 & 1.0005 & 0.9856 & $7.2\times10^{-5}$  & 0.9162 & 0.0373 \\
10  & 0.9744 & 1.0002 & 0.9907 & $6\times10^{-6}$    & 0.9695 & 0.0298 \\
12 & 0.9784 & 1.0001 & 0.9933 & $\lesssim 10^{-7}$  & 1.0131 & 0.0249 \\
14 & 0.9815 & 1.0000 & 0.9950 & $\lesssim 10^{-8}$  & 1.0501 & 0.0213 \\
\bottomrule
\end{tabular}
\caption{Comparison at $h^\star(1.5)$ for HQTFIM and at the exact
critical point $h_c^{\mathrm{std}}=1$ for the standard TFIM.
The hierarchical model remains strongly ordered at $h^\star$, whereas the
standard TFIM is critical with vanishing N\'eel order and increasing
entropy.}
\label{tab:critical-comparison}
\end{table}

A least-squares fit of $S_N^{\mathrm{std}}$ versus $\log N$ yields
\begin{equation}
  S_N^{\mathrm{std}} \approx (0.237\pm 0.01)\,\log N + \text{const},
\end{equation}
corresponding to an effective central charge
$c_{\rm eff}=3\times 0.237\approx 0.71$.
This overshoots the exact $c=1/2$, which is not surprising given the very
small sizes and the use of periodic boundary conditions; nevertheless the
data clearly support a logarithmic law.
By contrast, the HQTFIM entropy at $h^\star$ converges rapidly to one bit,
with no visible $\log N$ growth.

The order parameter further clarifies the physical regimes:
\begin{itemize}
\item At $h^\star$ the HQTFIM has $M^2_{\mathrm{stagg},N}\to 1$, consistent
with a deeply ordered phase.

\item At $h_c^{\mathrm{std}}$ the standard TFIM has
$M^2_{\mathrm{stagg},N}\to 0$, as expected at criticality.
\end{itemize}

\subsection{Exponents vs.\ $\alpha$ for the HQTFIM at $h^\star(\alpha)$}

For several values $\alpha\in\{1.2,1.5,2.0,2.5,3.0\}$ and sizes
$N\in\{4,6,8,10,12,14\}$ we compute $F_N$ and $M^2_{\mathrm{stagg},N}$ at
$h=h^\star(\alpha)$ and fit the finite-size behaviour to power laws
\begin{equation}
  1-F_N \sim N^{-\beta(\alpha)},\qquad
  1-M^2_{\mathrm{stagg},N} \sim N^{-\gamma(\alpha)}.
\end{equation}
The resulting exponents are reported in Table~\ref{tab:exponents-alpha}
and plotted in Fig.~\ref{fig:comparison} (bottom left and bottom middle).

\begin{table}[t]
\centering
\begin{tabular}{@{}cccccc@{}}
\toprule
$\alpha$ & $h^\star(\alpha)$ & $\beta(\alpha)$ & $R^2_\beta$ &
$\gamma(\alpha)$ & $R^2_\gamma$ \\
\midrule
1.2 & 0.5370 & 1.2224 & 0.972 & 2.1041 & 0.993 \\
1.5 & 0.4041 & 1.1335 & 0.984 & 1.9958 & 0.996 \\
2.0 & 0.2667 & 1.0484 & 0.996 & 1.8928 & 0.999 \\
2.5 & 0.1825 & 1.0180 & 0.999 & 1.8555 & 0.9999 \\
3.0 & 0.1270 & 1.0069 & 0.9997 & 1.8415 & 0.9998 \\
\bottomrule
\end{tabular}
\caption{Finite-size exponents for the HQTFIM at the RG
self-similarity field $h^\star(\alpha)$, obtained from log--log fits of
$1-F_N$ and $1-M^2_{\mathrm{stagg},N}$ versus $N$ for
$N\!\in\!\{4,6,8,10,12,14\}$.
Given the limited system sizes accessible, these exponents should be viewed
as effective finite-size exponents rather than asymptotic universal
quantities.}
\label{tab:exponents-alpha}
\end{table}

The exponents satisfy $\beta(\alpha)\approx 1$ and
$\gamma(\alpha)\approx 1.8$--$2.1$, with a mild drift as $\alpha$ is
varied.
Within the accessible range there is no hint of exponential convergence
in~$N$; instead the data are well described by power laws, which is
consistent with the interpretation of $h^\star(\alpha)$ as lying in an
ordered phase with algebraic corrections rather than at a critical point.

\subsection{Numerical phase diagram and true critical points of the HQTFIM}

Finally we investigate the \emph{true} critical field $h_c(\alpha)$ of the
HQTFIM.
For each $\alpha$ and fixed $N=12$ we sample
$M^2_{\mathrm{stagg},N}(h)$ on a uniform grid in $h\in[0.1,4.0]$, compute
the discrete derivative, and locate the point of steepest descent; we also
interpolate $M^2_{\mathrm{stagg},N}(h)$ and solve for
$M^2_{\mathrm{stagg},N}(h)=0.5$ when possible.

The resulting estimates are collected in
Table~\ref{tab:true-critical} and shown as phase diagrams in
Fig.~\ref{fig:comparison} (bottom right).

\begin{table}[t]
\centering
\begin{tabular}{@{}cccc@{}}
\toprule
$\alpha$ & $h^\star(\alpha)$ & $h_c(\alpha;N\!=\!12)$ (max drop) &
ratio $h_c/h^\star$ \\
\midrule
1.2 & 0.5370 & 2.70 & 5.0 \\
1.5 & 0.4041 & 3.00 & 7.4 \\
2.0 & 0.2667 & 3.30 & 12.4 \\
2.5 & 0.1825 & 3.50 & 19.2 \\
3.0 & 0.1270 & 3.60 & 28.3 \\
\bottomrule
\end{tabular}
\caption{Finite-size estimates of the true critical point $h_c(\alpha)$ of
the HQTFIM for $N=12$, obtained from the position of the steepest drop in
$M^2_{\mathrm{stagg},N}(h)$.
The hierarchical RG field $h^\star(\alpha)$ is much smaller and lies deep in
the ordered phase.}
\label{tab:true-critical}
\end{table}

Two robust conclusions follow:
\begin{itemize}
\item For all $\alpha$ studied, the true critical field $h_c(\alpha)$ is an
order of magnitude larger than $h^\star(\alpha)$; $h^\star$ is therefore
decisively \emph{not} a critical point but rather a special ordered-phase
point with a particularly clean N\'eel-cat structure.

\item The critical field $h_c(\alpha)$ drifts only mildly with $\alpha$
(increasing from $\sim 2.7$ to $\sim 3.6$ over the range
$\alpha\in[1.2,3.0]$), whereas $h^\star(\alpha)$ decreases rapidly.
\end{itemize}
A more refined finite-size scaling analysis in~$N$ would be required to
extrapolate $h_c(\alpha)$ to the thermodynamic limit, but the qualitative
separation between $h^\star(\alpha)$ and $h_c(\alpha)$ is already clear at
$N=12$.

\begin{figure}[p]
  \centering
  \includegraphics[width=0.95\textwidth]{hqtfim_vs_standard_comparison.png}
  \caption{Systematic comparison between the HQTFIM and the standard 1D
  TFIM.
  Top left:
  fidelity of the ground state with the N\'eel cat at fixed
  $h/J=0.404$ for both models.
  Top middle:
  entanglement entropy at the critical point of the standard TFIM
  ($h_c^{\mathrm{std}}$) and at $h^\star$ for the HQTFIM.
  Top right:
  staggered order parameter at the same points.
  Bottom left and middle:
  exponents $\beta(\alpha)$ and $\gamma(\alpha)$ at $h^\star(\alpha)$.
  Bottom right:
  phase diagrams $M^2_{\mathrm{stagg}}(h)$ for several $\alpha$ with dashed
  lines marking $h^\star(\alpha)$.}
  \label{fig:comparison}
\end{figure}

\section{Discussion: rigorous content, numerical evidence and open problems}

\subsection*{Mathematically rigorous content}

The following statements are fully rigorous and self-contained:

\begin{itemize}
\item Lemma~\ref{lem:block-spectrum} gives the exact spectrum and gap of
the two-spin block Hamiltonian~\eqref{eq:block-H}.

\item Theorem~\ref{thm:one-bit} proves that the N\'eel cat
$\ket{\psi_{\mathrm{cat}}}$ has exactly one bit of entanglement entropy for
any dyadic bipartition of a chain of length $N=2^H$.

\item Theorem~\ref{thm:TTN} constructs an explicit TTN representation of
$\ket{\psi_{\mathrm{cat}}}$ with bond dimension $\chi=2$ independent of the
system size.

\item The numerical tables and plots are obtained by straightforward exact
diagonalization followed by explicit evaluation of matrix elements and
reduced density matrices; the algorithms are standard and can be verified
directly from the supplied code.
\end{itemize}

\subsection*{Formal RG and phenomenological conclusions}

The block RG map of Proposition~\ref{prop:RG-map} is exact for a specific
formal coarse-graining scheme, but its relation to the rigorous
thermodynamic limit of the HQTFIM remains open; in particular, the
stability of the low-energy sector under iterated RG steps has not been
established analytically.
Accordingly, statements involving $h^\star(\alpha)$ as a distinguished field
must be regarded as phenomenological rather than fully proved.

\begin{conjecture}[N\'eel-cat ordered point at $h^\star(\alpha)$]
\label{conj:neel-cat}
For $\alpha>1$, at the RG self-similarity field $h=h^\star(\alpha)$ the
HQTFIM has long-range N\'eel order and its even-parity ground state satisfies
\[
  \lim_{N\to\infty} F_N = 1,\qquad
  \lim_{N\to\infty} S_N = 1\ \text{bit}
\]
for system sizes $N=2^H$.
\end{conjecture}

Within this framework, the numerics support the following picture:

\begin{itemize}
\item Numerical data are consistent with Conjecture~\ref{conj:neel-cat}.
At $h=h^\star(\alpha)$ the HQTFIM appears to be in a N\'eel-ordered phase,
with a ground state that approaches a N\'eel cat as $N$ increases.
This is supported by the increase of $F_N$, the approach of
$M^2_{\mathrm{stagg},N}$ to $1$, and the entropy $S_N$ rapidly approaching
one bit.

\item The rapidly vanishing even--odd splitting $\Delta_N$ is naturally
interpreted as finite-size tunnelling between symmetry-broken N\'eel
configurations, rather than as a bulk gap closing.

\item The power-law corrections $1-F_N\sim N^{-\beta(\alpha)}$ and
$1-M^2_{\mathrm{stagg},N}\sim N^{-\gamma(\alpha)}$ with
$\beta(\alpha)\approx 1$ and $\gamma(\alpha)\approx 2$ indicate algebraic
finite-size corrections at $h^\star(\alpha)$, not critical scaling in the
sense of a conformal point.

\item The true critical field $h_c(\alpha)$ of the HQTFIM is much larger
than $h^\star(\alpha)$ and only weakly dependent on~$\alpha$, whereas
$h^\star(\alpha)$ decreases strongly with~$\alpha$.
\end{itemize}

\subsection*{Open problems}

Several mathematically sharp questions remain:

\begin{itemize}
\item Prove the existence of a N\'eel-ordered phase for the HQTFIM for
$\alpha>1$ and locate the critical field $h_c(\alpha)$ in the
thermodynamic limit, possibly adapting rigorous techniques developed for
hierarchical and long-range classical models
\citep{bleher1975,bleher1988,bovier2006}.

\item Develop a rigorous multi-scale analysis of the HQTFIM resolvent
to control the projection onto the low-energy block subspace at each RG
step, in the spirit of Kato perturbation theory \citep{kato1995,simon2015},
and derive non-perturbative bounds on $1-F_N$ as $N\to\infty$.

\item Determine the entanglement scaling at the true critical point
$h_c(\alpha)$: does the hierarchical geometry modify the universality class
of the transition compared to the standard TFIM, and if so, how does the
entropy scale with~$N$?

\item Characterize the excitation spectrum above the N\'eel-cat sector at
$h^\star(\alpha)$, in particular the scaling of the bulk gap as a function
of $\alpha$ and~$h$.
\end{itemize}

\section*{Acknowledgments}

The author thanks the quantum information and mathematical physics
communities for discussions on hierarchical models, area laws and spectral
theory, and acknowledges the availability of open-source scientific
Python libraries that made the computations possible.

\bibliographystyle{plainnat}
\begin{thebibliography}{99}

\bibitem[Barouch and McCoy(1970)]{barouch1970}
E.~Barouch and B.~McCoy.
\newblock Statistical mechanics of the {XY} model. {II}. {S}pin-correlation
  functions.
\newblock {\em Phys.\ Rev.\ A}, 3:786--804, 1970.

\bibitem[Baker(1972)]{baker1972}
G.~A. Baker Jr.
\newblock Ising model with a scaling interaction.
\newblock {\em Phys.\ Rev.\ B}, 5:2622--2633, 1972.

\bibitem[Bleher and Sinai(1975)]{bleher1975}
P.~M. Bleher and Y.~G. Sinai.
\newblock Investigation of the critical point in models of the type of
  {D}yson's hierarchical models.
\newblock {\em Commun.\ Math.\ Phys.}, 45:247--278, 1975.

\bibitem[Bleher and Major(1988)]{bleher1988}
P.~M. Bleher and P.~Major.
\newblock Critical phenomena and constructions of {G}ibbs random fields.
\newblock {\em Russ.\ Math.\ Surv.}, 43:75--117, 1988.

\bibitem[Bovier(2006)]{bovier2006}
A.~Bovier.
\newblock {\em Statistical Mechanics of Disordered Systems}.
\newblock Cambridge University Press, 2006.

\bibitem[Calabrese and Cardy(2004)]{calabrese2004}
P.~Calabrese and J.~Cardy.
\newblock Entanglement entropy and quantum field theory.
\newblock {\em J.\ Stat.\ Mech.}, P06002, 2004.

\bibitem[Calabrese and Cardy(2009)]{calabrese2009}
P.~Calabrese and J.~Cardy.
\newblock Entanglement entropy and conformal field theory.
\newblock {\em J.\ Phys.\ A}, 42:504005, 2009.

\bibitem[Dyson(1969)]{dyson1969}
F.~J. Dyson.
\newblock Existence of a phase-transition in a one-dimensional Ising ferromagnet.
\newblock {\em Commun.\ Math.\ Phys.}, 12:91--107, 1969.

\bibitem[Hastings(2007)]{hastings2007}
M.~B. Hastings.
\newblock An area law for one-dimensional quantum systems.
\newblock {\em J.\ Stat.\ Mech.}, P08024, 2007.

\bibitem[Kato(1995)]{kato1995}
T.~Kato.
\newblock {\em Perturbation Theory for Linear Operators}.
\newblock Springer, 2nd edition, 1995.

\bibitem[Monthus(2018)]{monthus2018}
C.~Monthus.
\newblock {R}andom transverse field Ising model on the Dyson hierarchical lattice: 
  solution via a real space RG in configuration space.
\newblock {\em J.\ Stat.\ Mech.}, 033301, 2018.

\bibitem[Pfeuty(1970)]{pfeuty1970}
P.~Pfeuty.
\newblock The one-dimensional Ising model with a transverse field.
\newblock {\em Ann.\ Phys.}, 57:79--90, 1970.

\bibitem[Simon(2015)]{simon2015}
B.~Simon.
\newblock {\em Operator Theory}.
\newblock American Mathematical Society, 2015.

\bibitem[Vidal et~al.(2003)]{vidal2003}
G.~Vidal, J.~I. Latorre, E.~Rico, and A.~Kitaev.
\newblock Entanglement in quantum critical phenomena.
\newblock {\em Phys.\ Rev.\ Lett.}, 90:227902, 2003.

\end{thebibliography}

\end{document}
